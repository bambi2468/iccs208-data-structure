% You should title the file with a .tex extension (hw1.tex, for example)
\documentclass[a4paper, 11pt]{article}

\usepackage{amsmath}
\usepackage{amssymb}
\usepackage{fancyhdr}

\usepackage[margin=1in]{geometry}

\newcommand{\question}[2] {\vspace{.25in} \hrule\vspace{0.5em}
\noindent{\bf #1: #2} \vspace{0.5em}
\hrule \vspace{.10in}}
\renewcommand{\part}[1] {\vspace{.10in} {\bf (#1)}}

\newcommand{\myname}{Pimnara Klinniam}
\newcommand{\myemail}{pimnara.kln@student.mahidol.edu}
\newcommand{\myhwnum}{homework2}

\setlength{\parindent}{0pt}
\setlength{\parskip}{5pt plus 1pt}
 
\pagestyle{fancyplain}
\lhead{\fancyplain{}{\textbf{HW\myhwnum}}}      % Note the different brackets!
\rhead{\fancyplain{}{\myname\\ \myemail}}
\chead{\fancyplain{}{ICCS208}}

\begin{document}

\section*{Analysis of Fibonacci Array Growth}

The Fibonacci sequence is defined by
\[
F_{n+2} = F_{n+1} + F_n \quad \text{for } n \ge 0,
\]
with initial conditions \( F_1 = F_2 = 1 \).

\subsection*{Subtask I}

\begin{theorem}
For all \( n \ge 1 \),
\[
1 + F_1 + F_2 + \cdots + F_n = F_{n+2}.
\]
\end{theorem}

\begin{proof}
Prove by mathematical induction.

\textbf{Base case:} Let \( n = 1 \).
\[
1 + F_1 = 1 + 1 = 2,
\]
 \( F_3 = F_2 + F_1 = 1 + 1 = 2 \)

\textbf{Inductive hypothesis:} Assume that for some \( n \ge 1 \),
\[
1 + \sum_{k=1}^{n} F_k = F_{n+2}.
\]

\textbf{Inductive step:} show that
\[
1 + \sum_{k=1}^{n+1} F_k = F_{n+3}.
\]
Using the inductive hypothesis,
\[
1 + \sum_{k=1}^{n+1} F_k
= \left(1 + \sum_{k=1}^{n} F_k\right) + F_{n+1}
= F_{n+2} + F_{n+1}.
\]
By the Fibonacci recurrence relation,
\[
F_{n+3} = F_{n+2} + F_{n+1}.
\]
so,
\[
1 + \sum_{k=1}^{n+1} F_k = F_{n+3}.
\]
Therefore, the statement holds for all \( n \ge 1 \).
\end{proof}

\subsection*{Subtask II}

\begin{theorem}
For all \( n \ge 1 \),
\[
\frac{1}{F_n}\left(1 + \sum_{k=1}^{n} F_k\right) \le 3.
\]
\end{theorem}

\begin{proof}
By Theorem 1,
\[
1 + \sum_{k=1}^{n} F_k = F_{n+2}.
\]
Therefore,
\[
\frac{1}{F_n}\left(1 + \sum_{k=1}^{n} F_k\right)
= \frac{F_{n+2}}{F_n}.
\]
Using the Fibonacci recurrence,
\[
F_{n+2} = F_{n+1} + F_n = (F_n + F_{n-1}) + F_n = 2F_n + F_{n-1}.
\]
so,
\[
\frac{F_{n+2}}{F_n} = 2 + \frac{F_{n-1}}{F_n}.
\]
Since the Fibonacci sequence is increasing, we have
\[
\frac{F_{n-1}}{F_n} \le 1.
\]
Hence,
\[
\frac{F_{n+2}}{F_n} \le 3.
\]
\end{proof}

\subsection*{Subtask III}

We analyze the total number of copy operations when using the Fibonacci growth strategy for an ArrayList.

Initially, the array has capacity \( F_2 = 1 \). Whenever the array becomes full with capacity \( F_k \), it is resized to capacity \( F_{k+1} \), requiring \( F_k \) elements to be copied.

Assume that \( n = F_r + 1 \) for some integer \( r \ge 2 \). Then resizes occur for capacities
\[
F_2, F_3, \ldots, F_r.
\]
The total number of copy steps is therefore
\[
\sum_{k=2}^{r} F_k.
\]

By Theorem 1,
\[
\sum_{k=1}^{r} F_k = F_{r+2} - 1,
\]
which implies
\[
\sum_{k=2}^{r} F_k \le F_{r+2} - 1.
\]

From Subtask II, we know that
\[
F_{r+2} \le 3F_r.
\]
Since \( n = F_r + 1 \), it follows that \( F_r \le n \). Therefore,
\[
\sum_{k=2}^{r} F_k = O(n).
\]

Thus, inserting \( n \) elements using the Fibonacci growth scheme requires a total of \( O(n) \) copy operations, yielding an amortized cost of \( O(1) \) per insertion.
\qed

\end{document}
